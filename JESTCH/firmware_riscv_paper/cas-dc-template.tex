%% 
%% Copyright 2019-2020 Elsevier Ltd
%% 
%% This file is part of the 'CAS Bundle'.
%% --------------------------------------
%% 
%% It may be distributed under the conditions of the LaTeX Project Public
%% License, either version 1.2 of this license or (at your option) any
%% later version. The latest version of this license is in
%%    http://www.latex-project.org/lppl.txt
%% and version 1.2 or later is part of all distributions of LaTeX
%% version 1999/12/01 or later.
%% 
%% The list of all files belonging to the 'CAS Bundle' is
%% given in the file `manifest.txt'.
%% 
%% Template article for cas-dc documentclass for 
%% double column output.

%\documentclass[a4paper,fleqn,longmktitle]{cas-dc}
\documentclass[a4paper,fleqn]{cas-dc}

%\usepackage[authoryear,longnamesfirst]{natbib}
%\usepackage[authoryear]{natbib}
\usepackage[numbers,sort&compress]{natbib}

%%%Author definitions
\def\tsc#1{\csdef{#1}{\textsc{\lowercase{#1}}\xspace}}
\tsc{WGM}
\tsc{QE}
\tsc{EP}
\tsc{PMS}
\tsc{BEC}
\tsc{DE}
%%% 

\begin{document}
\let\ref\Cref 		
\let\eqref\Cref 	
\let\autoref\Cref 	
\let\WriteBookmarks\relax
\def\floatpagepagefraction{1}
\def\textpagefraction{.001}
\shorttitle{Loo Tung Lun et~al. / Engineering Science and Technology, an International Journal}
\footmarks{\url{https://doi.org/xx.xxxx/j.jestch.20xx.xx.xxx}\\
	2215-0986/\begingroup\tiny{©}\endgroup~20xx Karabuk University. Publishing services by Elsevier B.V.\\
	This is an open access article under the CC BY-NC-ND license (\url{http://creativecommons.org/licenses/by-nc-nd/4.0/}).
}

\bookmark[named = FirstPage]{A Simulated Firmware and Security Study on X86, ARM, RISC-V Platform and Security Enhancement For RISC-V} % Title bookmark used in the pdf
%**************** If the title is short, stay on the first line use [mode = short_title] otherwise ******************
%***************************************** use [mode = title] below ***************************************
\title [mode = title]{A Simulated Firmware Study on X86, ARM, RISC-V Platform and Security Enhancement For RISC-V}    

% Title mark notes if desired
%\tnotemark[1,2]

%\tnotetext[1]{This document is the results of the research
%   project funded by the National Science Foundation.}

%\tnotetext[2]{The second title footnote which is a longer text matter
%   to fill through the whole text width and overflow into
%   another line in the footnotes area of the first page.}

\author[1,2]{Loo, Tung Lun}
\address[1]{Sungai Petani, 08000 Kedah, Malaysia}

\author[2,2]{Mohamad Khairi Ishak}
\address[2]{USM, Malaysia}

\begin{abstract} 
In the era of Internet of Things, the number of connected embedded device went beyond
10 billion in 2020 and projected to hit 30 billion in 2025. All these active connected devices
create a stronger market demand to the servers in the cloud. In 2019, COVID pandemic hit
the world and caused a rise of demand for client devices, such as personal laptop and
desktops, for remote education purpose. A well-defined bootloader chosen upon different
ISA solution will ease the development process and shorten time to market, without
jeopardizing security. As such, a study on bootloaders architecture, solutions, and security
plays an important role in implementation the compute devices solutions today. There are
many well-known open-source bootloaders solutions available today with long
development and deployment history, such as UEFI/BIOS, Coreboot and Uboot. Recently,
RISC-V as an open-source Instruction Set Architecture also gain a lot of fame in products
creation and academic research purpose. In this paper, all Instruction Set Architecture boot
flow, boot solutions and their associated security activities are studied, summarized, and
experimented. A new proposed method to create a security block in Register Transfer Level
to generate Secure Hash Algorithms 5 digest is also implemented using Field
Programmable Gate Array. The tradeoff analysis here includes the numbers of logic
required and boot time penalty comparison of running Secure Hash Algorithms 5 in
bootloader and Register Transfer Level. With the proposed hardware implementation, it is
observed that there is significant performance boost compared to software execution.
\end{abstract}

% If any graphical abstract is needed
%\begin{graphicalabstract}
%\includegraphics{figs/grabs.pdf}
%\end{graphicalabstract}

% If any highlights is needed above the cover page
%\begin{highlights}
%\item Research highlights item 1
%\item Research highlights item 2
%\item Research highlights item 3
%\end{highlights}

% Article history - Should only be set by an editor
\received {27 July 2021}
\revised {xx Month 2021}
\accepted {xx Month 2021}
\online {xx Month 2021}

\begin{keywords}
RISC-V \sep 
Security \sep 
Firmware
\end{keywords}

\maketitle
\section{Introduction}
All compute devices today are powered by a few processors Instruction Set Architectures
(ISAs), predominantly x86, AMD, ARM, and MIPS which is later converged to RISC-V in
2021 (Jim Turley, EE Journal, 2021). These ISAs provide flexibilities and extensibilities to the
different engineering audiences, creating tremendous opportunities today that benefits
consumer in many custom applications and use cases, especially in the booming edge devices
in Internet of Things world.
While having multiple ISA options are good, it is often difficult to make a good decision on
which architecture to go for, because there are many factors that contribute to design decision.
Several key elements of consideration while picking an ISA are as below.
1. Time-To-Market (TTM)
The TTM factor is about how easy it is to enable an embedded system with collaterals
provided by the ISA provider. For example, the development time of an engineering team
(often called OEM/ODM) taking a new 11th Generation Intel chip and providing a full
solution with it. Several key factors that directly impact TTM are the availabilities of
documentation, system level open-source references and manufacturing technology.2
2. Cost
This factor includes cost of licensing, software, and hardware development cost that the
OEM/ODM needs to pay to get the products released.
3. Design flexibilities
The design flexibilities revolve around two key questions of “How easy it is to include a
new custom IP in a new design?” and “How easy it is to land firmware, driver, and
software support of a new IP?"


\section{ Problem Statements}
%\url{http://www.elsevier.com/locate/latex/} % Splitting is much convenient with \url than \href
The package is available at author resources page at Elsevier
\href{http://www.elsevier.com/locate/latex/}{\nolinkurl{http://www.elsevier.com/locate/latex/}}. % Allowing to break the url
The class may be moved or copied to a place, usually,
\verb+$TEXMF/tex/latex/elsevier/+, %$%%%%%%%%%%%%%%%%%%%%%%%%%%%%
or a folder which will be read by \LaTeX{} during document compilation. The \TeX{} file
database needs updation after moving/copying class file.  Usually,
we use commands like \verb+mktexlsr+ or \verb+texhash+ depending
upon the distribution and operating system. 

\subsection{Subsection of Installation}

\lipsum[2]

\subsubsection{Subsubsection of Installation}
\label{subsubsection}
\lipsum[2]

\paragraph{Installation}

\lipsum[2]

\section{Front matter}

The author names and affiliations could be formatted in two ways:
\begin{enumerate}[(1)]
\itemsep=-2pt 		% Separation between items
\itemindent=12pt 	% Extra indentation if necessary (12 pt matches with the text indentation)
\item Group the authors per affiliation.
\item Use footnotes to indicate the affiliations. 

\end{enumerate}
See the front matter of this document for examples. 
You are recommended to conform your choice to the journal you 
are submitting to. 
Several figure, section, table representation examples are given as: \ref{FIG:2}(a) and (b).
This is \ref{subsubsection}. 
Figure and table referencing can be given as \Cref{FIG:1} and \Cref{tbl1}.
Two consecutive figures can be written as \Cref{FIG:1,FIG:2} or \ref{FIG:1,FIG:2}.

\begin{figure}[t]
	\centering
	\includegraphics[scale=1]{figs/Fig1.pdf}
	\caption{Caption place holder.}
	\label{FIG:1}
\end{figure}

%\begin{figure*}
%	\centering
%	\begin{subfigure}[t]{0.5\textwidth}
%		\centering
%		\includegraphics[height=2in]{figs/Fig1.pdf}
%		\caption{}
%	\end{subfigure}%
%	~ 
%	\begin{subfigure}[t]{0.5\textwidth}
%		\centering
%		\includegraphics[height=2in]{figs/Fig1.pdf}
%		\caption{}
%	\end{subfigure}\\
%	\caption{Caption place holder.}
% \label{FIG:2}
%\end{figure*}

% Span subfigure example on double column
\begin{figure*}
	%\centering -> This is irrelevant because of the '.5\textwidth' as advised below.
	\begin{subfigure}[t]{.5\textwidth}\centering
		\includegraphics[width=.5\columnwidth]{figs/Fig1.pdf}
		\caption{}						% Put your sub-caption here if necessary 
	\end{subfigure}%
	\begin{subfigure}[t]{.5\textwidth}\centering
		\includegraphics[width=.5\columnwidth]{figs/Fig1.pdf}
		\caption{}						% Put your sub-caption here if necessary 
	\end{subfigure}
\caption{(a) First subfigure, and (b) second subfigure. Caption place holder. This is the subfigure landscape example on double column.}
\label{FIG:2}
\end{figure*}

% Sub figure example on single column
\begin{figure}[ht]
	\begin{subfigure}{.5\textwidth}
		\centering
		% include first image
		\includegraphics[width=.5\linewidth]{figs/Fig1.pdf}  
		\caption{}  					% Put your sub-caption here if necessary 
		\label{FIG:3a}
	\end{subfigure}
	\begin{subfigure}{.5\textwidth}
		\centering
		% include second image
		\includegraphics[width=.5\linewidth]{figs/Fig1.pdf}  
		\caption{} 						% Put your sub-caption here if necessary 
		\label{FIG:3b}
	\end{subfigure}
	\caption{Put your caption here. This is the subfigure example on single column.} 	% Put your caption here
	\label{FIG:3}
\end{figure}


\section{Bibliography styles}

There are various bibliography styles available. You can select the
style of your choice in the preamble of this document. These styles are
Elsevier styles based on standard styles like Harvard and Vancouver.
Please use Bib\TeX\ to generate your bibliography and include DOIs in URL
style whenever available.

Here are four sample references: 
\cite{Fortunato2010}
\cite{Fortunato2010,NewmanGirvan2004} 	
\cite{Fortunato2010,Vehlowetal2013} 	
\cite{Fortunato2010,NewmanGirvan2004,Vehlowetal2013,Chenetal2013,Clausetetal2004,FabricioLiang2013,Gregory2011,FortunatoBarthelemy2007,HullermeierRifqi2009,Nepuszetal2008,Newman2013}. 					

\section{Floats}
{Figures} may be included using the command,\linebreak 
\verb+\includegraphics+ in
combination with or without its several options to further control
graphic. \verb+\includegraphics+ is provided by {graphic[s,x].sty}
which is part of any standard \LaTeX{} distribution. 
{graphicx.sty} is loaded by default. \LaTeX{} accepts figures in
the \linebreak postscript format while pdf\LaTeX{} accepts {*.pdf},
{*.mps} (metapost), {*.jpg} and {*.png} formats. 
pdf\LaTeX{} does not accept graphic files in the postscript format. 

\begin{figure}
	\centering
	\includegraphics[scale=1]{figs/Fig1.pdf}
	\caption{The evanescent light - $1S$ quadrupole coupling
	($g_{1,l}$) scaled to the bulk exciton-photon coupling
	($g_{1,2}$). The size parameter $kr_{0}$ is denoted as $x$ and
	the \PMS is placed directly on the cuprous oxide sample ($\delta
	r=0$, See also \protect\ref{FIG:3}).}
	\label{FIG:4}
\end{figure}

% Span figure example on double column
\begin{figure*}
	\centering
	\includegraphics[width=\textwidth,height=2in]{figs/Fig2.pdf}
	\caption{Schematic of formation of the evanescent polariton on
		linear chain of \PMS. The actual dispersion is determined by 
		the ratio of two coupling parameters such as exciton-\WGM coupling
		and \WGM-\WGM coupling between the microspheres.}
	\label{FIG:5}
\end{figure*}

The \verb+table+ environment is handy for marking up tabular
material. If users want to use {multirow.sty},
{array.sty}, etc., to fine control/enhance the tables, they
are welcome to load any package of their choice and
{cas-dc.cls} will work in combination with all loaded
packages.
% Single column table example
\begin{table}[width=1\linewidth,cols=4,pos=t]
\caption{This is a test caption. This is a test caption. This is a test
caption. This is a test caption.}\label{tbl1}
\begin{tabular*}{\tblwidth}{@{} LLLL@{} }
\toprule
Col 1 & Col 2 & Col 3 & Col4\\
\midrule
12345 & 12345 & 123 & 12345 \\
12345 & 12345 & 123 & 12345 \\
12345 & 12345 & 123 & 12345 \\
12345 & 12345 & 123 & 12345 \\
12345 & 12345 & 123 & 12345 \\
\bottomrule
\end{tabular*}
\end{table}

% Span table example on double column
\begin{table*}[width=1.0\textwidth,cols=4,pos=h]
	\caption{This is a test caption. This is a test caption. This is a test caption. This is a test caption.}
	\begin{tabular*}{\tblwidth}{@{} LLLLLLL@{} }
		\toprule
		Col 1 & Col 2 & Col 3 & Col4 & Col5 & Col6 & Col7\\
		\midrule
		12345 & 12345 & 123 & 12345 & 123 & 12345 & 123 \\
		12345 & 12345 & 123 & 12345 & 123 & 12345 & 123 \\
		12345 & 12345 & 123 & 12345 & 123 & 12345 & 123 \\
		12345 & 12345 & 123 & 12345 & 123 & 12345 & 123 \\
		12345 & 12345 & 123 & 12345 & 123 & 12345 & 123 \\
		\bottomrule
	\end{tabular*}
\end{table*}

\section{Theorem and theorem like environments} % \section[Theorem and ...]{Theorem and theorem like environments}

{cas-dc.cls} provides a few shortcuts to format theorems and
theorem-like environments with ease. In all commands, the options that
are used with the \verb+\newtheorem+ command will work exactly in the same
manner. {cas-dc.cls} provides three commands to format theorem or
theorem-like environments: 

\begin{verbatim}
 \newtheorem{theorem}{Theorem}
 \newtheorem{lemma}[theorem]{Lemma}
 \newdefinition{rmk}{Remark}
 \newproof{pf}{Proof}
 \newproof{pot}{Proof of Theorem \ref{thm2}}
\end{verbatim}

The \verb+\newtheorem+ command formats a
theorem in \LaTeX's default style with italicized font, bold font
for theorem heading and theorem number at the right hand side of the
theorem heading. It also optionally accepts an argument which
will be printed as an extra heading in parentheses. 

\begin{verbatim}
  \begin{theorem} 
   For system (8), consensus can be achieved with 
   $\|T_{\omega z}$ ...
     \begin{eqnarray}\label{10}
     ....
     \end{eqnarray}
  \end{theorem}
\end{verbatim}  

\newtheorem{theorem}{Theorem}

\begin{theorem}
For system (8), consensus can be achieved with 
$\|T_{\omega z}$ ...
\begin{eqnarray}\label{10}
	\lambda_{1S}/2 \pi \left({\epsilon_{Cu2O}-1}\right)^{1/2} = 414 \mbox{
	\AA} \gg a_B = 4.6 \mbox{ \AA}  
\end{eqnarray}
\end{theorem}

The \verb+\newdefinition+ command is the same in
all respects as its \verb+\newtheorem+ counterpart except that
the font shape is roman instead of italic. Both
\verb+\newdefinition+ and \verb+\newtheorem+ commands
automatically define counters for the environments defined.

The \verb+\newproof+ command defines proof environments with
upright font shape. No counters are defined. 

% See the \newtheorem example below:
\begin{theorem}\label{thm}
	The \WGM evanescent field penetration depth into the cuprous oxide
	adjacent crystal is much larger than the \QE radius: 
	
	\begin{equation*} % Equation without number (needs *)
	\lambda_{1S}/2 \pi \left({\epsilon_{Cu2O}-1}\right)^{1/2} = 414 \mbox{
		\AA} \gg a_B = 4.6 \mbox{ \AA}  
	\end{equation*}
	
\end{theorem}

% See the \newdefinition example below:
\newdefinition{definition}{Definition}
\begin{definition}
	The bulk and evanescent polaritons in cuprous oxide
	are formed through the quadrupole part of the light-matter
	interaction:
	
	\begin{equation*}
	H_{int} = \frac{i e }{m \omega_{1S}} {\bf E}_{i,s} \cdot {\bf p}
	\end{equation*}
	
\end{definition}
%
\begin{equation} 	% Equation with number
\lambda_{1S}/2 \pi \Big({\epsilon_{Cu2O}-1}\Big)^{1/2} = 414 \mbox{
	\AA} \gg a_B = 4.6 \mbox{ \AA}  
\label{Eq.2}
\end{equation}

\begin{equation}
y_{t} = \phi_{1} y_{t-1} + \epsilon_{t}
\label{Eq.3}
\end{equation}

% Subequation example
\begin{subequations}
\setlength{\jot}{6pt} % Spacing between equations in the group
	\begin{gather}
	R_0 = 0 \label{Eq.4a}\\
	N_0 = 0 \label{Eq.4b}
	\end{gather}
	\label{Eq.4}
\end{subequations}
%
\begin{equation}
D\left(C_{A},C_{B}\right) = \min X_{A}\in C_{A},X_{B}\in C_{B} 
d\left(X_{A},X_{B}\right)
\label{Eq.5}
\end{equation}

%\begin{equation}
%\setlength{\jot}{6pt} % Spacing between equations in the group
%\begin{aligned}
%T_{P} &= K_{T}. \rho . n^{2}_{p} . D^{4}_{p} \\
%Q_{P} &= K_{Q}. \rho . n^{2}_{p} . D^{5}_{p} \\
%N_{P} &= K_{N}. \rho . n^{2}_{p} . D^{6}_{p} \\
%K_{P} &= K_{K}. \rho . n^{2}_{p} . D^{7}_{p}
%\end{aligned}
%\end{equation}

\begin{equation}
y_{t} = \phi_{1} y_{t-1} + \epsilon_{t}
\label{Eq.6}
\end{equation}
% For proper spacing add % before and/or after equations
% \newproof command helps to define proof and custom proof environments without counters as provided in the example code. % Given below is an example of proof of theorem kind. 
\newproof{pot}{Proof of \ref{thm}}
\begin{pot}
	The photon part of the polariton trapped inside the \PMS moves as
	it would move in a micro-cavity of the effective modal volume $V
	\ll 4 \pi r_{0}^{3} /3$. Consequently, it can escape through the
	evanescent field. This evanescent field essentially has a quantum
	origin and is due to tunneling through the potential caused by
	dielectric mismatch on the \PMS surface. Therefore, we define the
	\emph{evanescent} polariton (\EP) as an evanescent light - \QE
	coherent superposition as in Eq. \eqref{Eq.2}. Eq. \eqref{Eq.6} can be referenced in this form. Multiple equations can be represented as in Eqs. \eqref{Eq.5,Eq.6}. Multiple consecutive equations can be shown as Eqs. \eqref{Eq.3,Eq.4,Eq.5,Eq.6}.
\end{pot}

\section{Enumerated and Itemized Lists} % \section[Enumerated ...]{Enumerated and Itemized Lists}
{cas-dc.cls} provides an extended list processing macros
which makes the usage a bit more user friendly than the default
\LaTeX{} list macros. With an optional argument to the
\verb+\begin{enumerate}+ command, you can change the list counter
type and its attributes. If you would like to use classical enumeration/itemize styles,
you may comment out "Customized Enumeration" section in the cas-common.sty file
and use \verb+\usepackage{enumerate}+ or\linebreak \verb+\usepackage{enumitem}+ that can be added to the cas-dc.cls file instead.

\begin{verbatim}
 \begin{enumerate}[1.]
 \item The enumerate environment starts with an optional
   argument `1.', so that the item counter will be suffixed
   by a period.
 \item You can use `a)' for alphabetical counter and '(i)' 
  for roman counter.
  \begin{enumerate}[a)]
    \item Another level of list with alphabetical counter.
    \item One more item before we start another.
    \item One more item before we start another.
    \item One more item before we start another.
    \item One more item before we start another.
\end{verbatim}

Further, the enhanced list environment allows one to prefix a
string like `step' to all the item numbers.  

\begin{verbatim}
 \begin{enumerate}[Step 1.]
  \item This is the first step of the example list.
  \item Obviously this is the second step.
  \item The final step to wind up this example.
 \end{enumerate}
\end{verbatim}

% See the \begin{enumerate example below:
\begin{enumerate}[(1)]
	\item The enumerate environment starts with an optional
	argument ‘1.’ so that the item counter will be suffixed
	by a period as in the optional argument.
	\item If you provide a closing parenthesis to the number in the
	optional argument, the output will have closing parenthesis
	for all the item counters.
	\item You can use ‘(a)’ for alphabetical counter and ’(i)’ for
	roman counter.
	\begin{enumerate}[a)]
		\item Another level of list with alphabetical counter.
		\item One more item before we start another.
		\begin{enumerate}[(i)]
			\item This item has roman numeral counter.
			\item Another one before we close the third level.
		\end{enumerate}
		\item Third item in second level.
	\end{enumerate}
	\item All list items conclude with this step.
\end{enumerate}

\section{Cross-references}
In electronic publications, articles may be internally 
hyperlinked. Hyperlinks are generated from proper cross-references in the article. For example, the words \textcolor{black!80}{Fig.~1} will never be more than simple text,
whereas the proper cross-reference \verb+\ref{tiger}+ or \verb+\Cref{tiger}+ may be
turned into a hyperlink to the figure itself:
Fig.~1. In the same way,
the words Ref.~[1] will fail to turn into a
hyperlink; the proper cross-reference is \verb+\cite{Knuth96}+.
Cross-referencing is possible in \LaTeX{} for sections,
subsections, formulae, figures, tables, and literature
references.

\section{Bibliography}

Two bibliographic style files (\verb+*.bst+) are provided ---
{model1-num-names.bst}, {model2-names.bst}, and {elsarticle-num.bst} --- the first one can be
used for the numbered scheme. This can also be used for the numbered
with new options of {natbib.sty}. The second one is for the author year
scheme. When you use model2-names.bst, the citation commands will be
like \verb+\citep+,  \verb+\citet+, \verb+\citealt+ etc. However when
you use model1-num-names.bst, you may use only \verb+\cite+ command. The third one is 
used in this template which is resembling to the final layout.

\verb+thebibliography+ environment. Each reference is a \linebreak
\verb+\bibitem+ and each \verb+\bibitem+ is identified by a label,
by which it can be cited in the text:

\noindent In connection with cross-referencing and
possible future hyperlinking, it is not a good idea to collect
more that one literature item in one \verb+\bibitem+. The
so-called Harvard or author-year style of referencing is enabled
by the \LaTeX{} package {natbib}. With this package the
literature can be cited as follows:

\begin{itemize}
%\begin{enumerate}[\textbullet]
\itemsep=-1pt 		% Adjusting item separation
\itemindent=-3pt 	% Extra indentation from the left margin if necessary
\item Parenthetical: \verb+\citep{WB96}+ produces (Wettig \& Brown, \break 1996).
\item Textual: \verb+\citet{ESG96}+ produces Elson et al. (1996).
\item An affix and part of a reference: %\break 
\verb+\citep[e.g.][Ch. 2]{Gea97}+ produces (e.g. Governato et
al., 1997, Ch. 2).
%\end{enumerate}
\end{itemize}

In the numbered scheme of citation, \verb+\cite{<label>}+ is used,
since \verb+\citep+ or \verb+\citet+ has no relevance in the numbered
scheme. {natbib} package is loaded by {cas-dc} with
\verb+numbers+ as default option. You can change this to author-year
or harvard scheme by adding option \verb+authoryear+ in the class
loading command. If you want to use more options of the {natbib}
package, you can do so with the \verb+\biboptions+ command. For
details of various options of the {natbib} package, please take a
look at the {natbib} documentation, which is part of any standard
\LaTeX{} installation.

%\appendix
%\section{My Appendix}
%Appendix sections are coded under \verb+\appendix+.

%\verb+\printcredits+ command is used after appendix sections to list 
%author credit taxonomy contribution roles tagged using \verb+\credit+ 
%in frontmatter.

\section{Conclusion}

\lipsum[2]

\setcounter{equation}{0} % Reset the equation counter to 1 for Appendix
\renewcommand{\theequation}{A.\arabic{equation}} % Modify the equation for Appendix A
\pdfbookmark[section]{Appendix A}{} % Include bookmark in the pdf (write the same name from the section below)
\section*{Appendix A}
Appendix sections are coded under \verb+\appendix+.

\verb+\printcredits+ command is used after appendix sections to list 
author credit taxonomy contribution roles tagged using \verb+\credit+ 
in frontmatter.

\pdfbookmark[subsection]{A.1 Subsection of appendix a}{} % Include bookmark in the pdf (write the same name from the Abstract subsection below)
\subsection*{A.1 Subsection of appendix a}

\lipsum[2] Eq. \eqref{Eq.A1} can be referenced in this form in the Appendix A.

\begin{equation}
y_{t} = \phi_{1} y_{t-1} + \epsilon_{t}
\label{Eq.A1}
\end{equation}

\pdfbookmark[subsubsection]{A.1.1 Subsubsection of appendix a}{} % Include bookmark in the pdf (write the same name from the Abstract subsection below)
\subsubsection*{A.1.1 Subsubsection of appendix a}

\lipsum[4]

\pdfbookmark[section]{Declaration of competing interest / Conflict of interest}{} % Include bookmark in the pdf (write the same name from the section below)
\section*{Declaration of competing interest / Conflict of interest}
The authors declare that they have no known competing financial
interests or personal relationships that could have appeared
to influence the work reported in this paper.
%The authors declare no conflict of interest.

\pdfbookmark[section]{Acknowledgment}{} % Include bookmark in the pdf (write the same name from the section below)
\section*{Acknowledgment}
The authors would like to thank the editors and anonymous
reviewers for providing insightful suggestions and comments
to improve the quality of research paper.

\pdfbookmark[section]{CRediT authorship contribution statement}{} % Include bookmark in the pdf (write the same name from the section below)
\printcredits

% It is suggested to add the DOI of the each possible reference using the url site style as in the given example.
\pdfbookmark[section]{References}{} % Include bookmark in the pdf
\hyphenpenalty=10000 % Almost no hypenation in biblio (higher value means less hypenation)
%% Loading bibliography style file
%\bibliographystyle{model1-num-names}
%\bibliographystyle{cas-model2-names}
\bibliographystyle{elsarticle-num} 				
% Loading bibliography database
\bibliography{cas-refs}
%\endgroup

\vskip6pt

\end{document}

